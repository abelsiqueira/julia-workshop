\section{Introdução}

\myframesec{
  \begin{itemize}
    \item<1-> Boa linguagem inicial;
    \item<2-> Boa linguagem para pesquisadores das ciências exatas;
    \item<3-> Almeja substituir o MatLab (sintaxe similar);
    \item<4-> Usa coisas boas de outras linguagens (C, Fortran, MatLab,
      Python,\ldots);
    \item<5-> Código aberto, ativamente desenvolvida;
    \item<6-> Muitas funções prontas;
    \item<7-> Fácil de instalar pacotes novos;
    \item<8-> Multiplataforma.
  \end{itemize}
}

\subsection{Interface}

\myframesec{
  \begin{itemize}
    \item<1-> Diferente do MatLab; Parece com C/Fortran; Muito similar à Python;
    \item<2-> \cmdinline{julia arquivo.jl} roda o arquivo;
    \item<3-> \cmdinline{julia} abre o REPL (Read-Eval-Print loop);
    \item<4-> Não tem interface gráfica embutida (Veja Atom/Juno e Jupyter)
  \end{itemize}
}

\subsection{Pacotes}

\myframesec{
  \begin{itemize}
    \item<1-> Vem com funções matemáticas básicas e mais avançadas, como Cholesky;
    \item<2-> Diferente do MatLab, não é tudo embutido;
    \item<3-> Muitas coisas são implementadas e indexadas, sendo facilmente
      instaladas com
      \cmdinline{Pkg.add("Pacote")};
    \item<4-> Alguns pacotes não são indexados, mas podem ser instalados com
      \cmdinline{Pkg.clone("link/para/github")}
  \end{itemize}
}

\subsection{Velocidade}

\myframesec{
  \begin{itemize}
    \item<1-> A primeira execução compila o código (então é lenta);
    \item<2-> As próximas execuções são consideravelmente rápidas;
    \item<3-> Você pode chamar funções de C e Fortran, migrando as partes lentas
      para uma ``linguagem mais rápida''.
  \end{itemize}
}

\subsection{É usado na área}

\myframesec{
  \begin{itemize}
    \item<1-> {\tt JuliaOpt} - pacotes relacionado à otimização;
      3 artigos à respeito; 11 artigos citando; usado em vários cursos;
    \item<2-> Alguns professores e pós-doutorandos;
    \item<3-> Vários alunos de doutorado.
  \end{itemize}
}
